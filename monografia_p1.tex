\documentclass[]{politex}

% ========== Packages ==========
\usepackage[utf8]{inputenc}
\usepackage{amsmath,amsthm,amsfonts,amssymb}
\usepackage{graphicx,cite,enumerate}

% ========== Language options ==========
\usepackage[brazil]{babel}
%\usepackage[english]{babel}


% ========== ABNT (requer ABNTeX 2) ==========
%	http://www.ctan.org/tex-archive/macros/latex/contrib/abntex2
%\usepackage[num]{abntex2cite}

% Forçar o abntex2 a usar [ ] nas referências ao invés de ( )
%\citebrackets{[}{]}


% ========== Lorem ipsum ==========
\usepackage{blindtext}

% ========== Opções do documento ==========
% Título
\titulo{Desenvolvimento de dispositivos eletrônicos para monitoramento de qualidade e conforto em ambientes empresariais}

% Autor
\autor{Isabella Bologna Salomão\\%
		Renato de Oliveira Freitas}


% Orientador / Coorientador
\orientador{Prof. Dr. Gustavo P. Rehder\\%
			Prof.ª Dra. Cíntia Borges Margi}
%\coorientador{Eng. Conrado Leite de Vitor (empresa PullUp)}

% Tipo de documento
\tcc{Eletricista com ênfase em Eletrônica e Sistemas}
%\dissertacao{Engenharia Elétrica}
%\teseDOC{Engenharia Elétrica}
%\teseLD
%\memorialLD

% Departamento e área de concentração
\departamento{PSI - Eletrônica e Sistemas}
%\areaConcentracao{Área de concentração}

% Local
\local{São Paulo}

% Ano
\data{2020}


\begin{document}
% ========== Capa e folhas de rosto ==========
\capa
%\falsafolhaderosto
\folhaderosto


% ========== Resumo ==========
\begin{resumo}
Resumo...
%
\\[3\baselineskip]
%
\textbf{Palavras-Chave} -- Internet of Things, Conforto Térmico, Conforto Acústico, Conforto Luminoso, Wireless Sensor Network, Green Buildings, Smart Office.
\end{resumo}


% ========== Sumário ==========
\sumario


% ========== Elementos textuais ==========

\part{Introdução}

IoT blablabla monitorar conforto balbla ambientes empresariais

\chapter{Declaração da Necessidade}
% Contexto, PCC

Com o aumento do tempo que as pessoas passam em ambientes fechados, como escritórios, há também nos últimos anos um crescente interesse por monitorar esses ambientes, garantindo não só saúde e conforto para as pessoas, mas também podendo atuar de maneira energeticamente sustentável. Esses espaços são comumente chamados de prédios inteligentes (\textit{smart buildings}, do inglês) ou \textit{green buildings} (em português, construção sustentável) \cite{GreenBuildings} \cite{EnergyBuildings}. 

Não apenas o monitoramento dos ambientes, mas torna-se necessário, no desenvolvimento de construções sustentáveis, que seja pensado na automação dos edifícios desde o projeto e sua concepção, ocorrendo de forma integrada à construção civil. Isto ocorre pois com uma pesquisa mais aprofundada no conforto dos ambientes pode interferir no projeto, sendo repensados materiais utilizados, além de aquecimento, ventilação, iluminação, dentre outros. 

Foi com essa necessidade e a proposta de desenvolver um dispositivo eletrônico, que o professor Vanderley Moacyr, do departamento de Construção Civil da Poli (PCC) entrou em contato. A ideia é que seja desenvolvido um dispositivo capaz de fazer medições de parâmetros relacionados ao conforto nos ambientes internos de uma construção e também coletar dados das pessoas ali presentes, para assim saber o real impacto dos indicadores de conforto. Além disso, é importante que os dispositivos possam estar integrados a uma central de controle, que possa monitorar todo o ambiente. 

Assim, a construção de uma rede de dispositivos sensoreados tem, além de uma aplicação prática monitorando a qualidade para as pessoas, também grande utilidade em pesquisas de construção civil e arquitetura, com medições mais precisas e incluindo um elemento muitas vezes deixado de lado: o fator humano. 

Em edifícios, escritórios são hoje os que ocupam a maior área física e tem o maior consumo de energia, sendo sistemas de iluminação, aquecimento e resfriamento (como ar condicionados) os principais causadores do alto consumo\cite{EnergyBuildings}. Por isso, escritórios são o nicho escolhido para o desenvolvimento dessa rede de dispositivos, podendo ser testada nas salas do departamento de Construção Civil. 


% Dificuldade em estudar elementos relacionados a conforto ?
% Falar do que já existe? 

\chapter{Descrição do Problema}
% O que monitorar?

% Apesar de a maioria das soluções existentes focar em conforto térmico \cite{}, ou em qualidade do ar\cite{}, 

%% Aqui ou na pesquisa do tema?
O conforto e a qualidade em ambientes internos é determinado através de quatro principais indicadores: térmico, acústico, luminoso e olfativo\cite{ComfortBox}. 

A fim de conseguirmos monitorar esses indicadores, é necessário medirmos diversos dados a respeito do ambiente a ser estudado: %%% ??  essa frase
\begin{itemize}
\item Térmico: temperatura ambiente e umidade relativa
\item Acústico: ruído ambiente
\item Luminoso: intensidade e temperatura da luz incidente
\item Olfativo: VOC
\end{itemize}

Além da medição de VOC, para uma boa qualidade do ar é também importante medirmos a concentração de CO2 \cite{}

Não apenas esses elementos são importantes, mas também a combinação deles afeta a percepção de conforto pelas pessoas \cite{ComfortOffice}. Assim, faz-se mais necessário que haja uma medição completa dos elementos presentes no ambiente a ser estudado. 



\chapter{Objetivos} 
% arvore de objetivos

\part{Conforto em Ambientes Internos} % Tema? ~Estado da Arte~

\chapter{Indicadores de Conforto} %?

Ainda que conforto seja um termo qualitativo e subjetivo, existem regulamentações e normas, além de estudos, que podem nos dar critérios quantitativos para analisar os dados medidos, que serão usados como % margem inicial? warnings de qualidade sendo afetada?

A legislação brasileira determina os valores máximos e mínimos dos indicadores de conforto no ambiente para que haja boas condições de trabalho: 

\begin{citacaoLonga}
\textbf{NR17 do Ministério do Trabalho} \cite{NR17}

17.5. Condições ambientais de trabalho.

17.5.2. Nos locais de trabalho onde são executadas atividades que exijam solicitação intelectual e atenção constantes, tais como: salas de controle, laboratórios, escritórios, salas de desenvolvimento ou análise de projetos, dentre outros, são recomendadas as seguintes condições de conforto:

a) níveis de ruído de acordo com o estabelecido na NBR 10152, norma brasileira registrada no INMETRO;

b) índice de temperatura efetiva entre 20oC (vinte) e 23oC (vinte e três graus centígrados);

[...]

d) umidade relativa do ar não inferior a 40 (quarenta) por cento.

17.5.2.1. Para as atividades que possuam as características definidas no subitem 17.5.2, mas não apresentam equivalência ou correlação com aquelas relacionadas na NBR 10152, o nível de ruído aceitável para efeito de conforto será de até 65 dB (A)

[...]

17.5.3.3. Os níveis mínimos de iluminamento a serem observados nos locais de trabalho são os valores de iluminâncias estabelecidos na NBR 5413, norma brasileira registrada no INMETRO.
\end{citacaoLonga}

\begin{citacaoLonga}
\textbf{NBR 10152} \cite{NBR10152} para Escritórios

Salas de reunião: 30 - 40 dB(A)

Salas de gerência, Salas de projetos e de administração: 35 - 45 dB(A)

Salas de computadores: 45 - 65 dB(A)

Salas de mecanografia: 50 - 60 dB(A)

\textbf{NBR 5413} \cite{NBR5413}

Para escritórios: 500 - 750 - 1000 lux
\end{citacaoLonga}

Alguns outros índices, como VOC e CO2 não são descritos na legislação. 

Conforto Visual

Além da intensidade da luz incidente, cujos níveis são estabelecida na legislação, a temperatura da cor da luz incidente também influencia tanto na saúde quanto no conforto e na atenção das pessoas, como visto em \cite{VisualComfort}: 
\begin{itemize}
\item Conforto, luz natural: 3000K - 6000K
\item Concentração: acima de 5300K 
\end{itemize}
A temperatura também relaciona-se com a qualidade do ambiente. A muitos anos sabe-se que a luz azul emitida, de maior temperatura, causa danos à retina \cite{BlueLight}. 
% CO2 e VOC

\chapter{Benchmark} % ?? 




% ========== Referências ==========
% --- IEEE ---
%	http://www.ctan.org/tex-archive/macros/latex/contrib/IEEEtran
%\bibliographystyle{IEEEbib}

% --- ABNT (requer ABNTeX 2) ---
%	http://www.ctan.org/tex-archive/macros/latex/contrib/abntex2
%\bibliographystyle{abntex2-num}
\bibliographystyle{plain}

\bibliography{reference_p1}

%L. Ciabattoni, F. Ferracuti, G. Ippoliti, S. Longhi and G. Turri, "IoT based indoor personal comfort levels monitoring," 2016 IEEE International Conference on Consumer Electronics (ICCE), Las Vegas, NV, 2016, pp. 125-126, doi: 10.1109/ICCE.2016.7430548.


\end{document}
