\documentclass[]{politex}

% ========== Packages ==========
\usepackage[utf8]{inputenc}
\usepackage{amsmath,amsthm,amsfonts,amssymb}
\usepackage{graphicx,cite,enumerate}
\graphicspath{ {./images/} }

% ========== Language options ==========
\usepackage[brazil]{babel}
%\usepackage[english]{babel}


% ========== ABNT (requer ABNTeX 2) ==========
%	http://www.ctan.org/tex-archive/macros/latex/contrib/abntex2
%\usepackage[num]{abntex2cite}

% Forçar o abntex2 a usar [ ] nas referências ao invés de ( )
%\citebrackets{[}{]}


% ========== Lorem ipsum ==========
\usepackage{blindtext}

% ========== Opções do documento ==========
% Título
\titulo{Desenvolvimento de dispositivos eletrônicos para monitoramento de qualidade e conforto em ambientes empresariais}

% Autor
\autor{Isabella Bologna Salomão\\%
		Renato de Oliveira Freitas}


% Orientador / Coorientador
\orientador{Prof. Dr. Gustavo P. Rehder\\%
			Prof.ª Dra. Cíntia Borges Margi}
%\coorientador{Eng. Conrado Leite de Vitor (empresa PullUp)}

% Tipo de documento
\tcc{Eletricista com ênfase em Eletrônica e Sistemas}
%\dissertacao{Engenharia Elétrica}
%\teseDOC{Engenharia Elétrica}
%\teseLD
%\memorialLD

% Departamento e área de concentração
\departamento{PSI - Eletrônica e Sistemas}
%\areaConcentracao{Área de concentração}

% Local
\local{São Paulo}

% Ano
\data{2020}


\begin{document}
% ========== Capa e folhas de rosto ==========
\capa
%\falsafolhaderosto
\folhaderosto


% ========== Resumo ==========
\begin{resumo}
Resumo...
%
\\[3\baselineskip]
%
\textbf{Palavras-Chave} -- Internet of Things, Conforto Térmico, Conforto Acústico, Conforto Luminoso, Qualidade do Ar, Wireless Sensor Network, Green Buildings, Smart Office.
\end{resumo}


% ========== Sumário ==========
\sumario


% ========== Elementos textuais ==========

\part{Introdução}

\chapter{Declaração da Necessidade}
% Contexto, PCC

Com o aumento do tempo que as pessoas passam em ambientes fechados, como escritórios, há também nos últimos anos um crescente interesse por monitorar esses ambientes, garantindo não só saúde e conforto para as pessoas, mas também sua produtividade, podendo até mesmo atuar de maneira energeticamente sustentável. Esses espaços são comumente chamados de prédios inteligentes (\textit{smart buildings}, do inglês), e dentro do contexto sustentável, essa automação é importante para os \textit{green buildings} (em português, construções sustentáveis) \cite{GreenBuildings} \cite{EnergyBuildings}. 

Não apenas o monitoramento dos ambientes, mas torna-se necessário, no desenvolvimento de construções sustentáveis, que seja pensado na automação dos edifícios desde o projeto e sua concepção, ocorrendo de forma integrada à construção civil. Isto ocorre pois com uma pesquisa mais aprofundada no conforto dos ambientes pode interferir no projeto, sendo repensados materiais utilizados, assim como sistemas de aquecimento, ventilação, iluminação, dentre outros. 

Foi com essa necessidade e a proposta de desenvolver um dispositivo eletrônico, que o professor Vanderley M. John, do departamento de Construção Civil da Poli (PCC) e coordenador do CICS (Centro de Inovação em Construção Sustentável da USP)\cite{CICS}, entrou em contato. A ideia é que seja desenvolvido um dispositivo capaz de fazer medições de parâmetros relacionados ao conforto nos ambientes internos de uma construção, coletando também a opinião das pessoas ali presentes, para assim saber o real impacto dos indicadores de conforto. Além disso, é importante que os dispositivos possam estar integrados a uma central, que possa analisar e monitorar todo o ambiente. 

Assim, a construção de uma rede de dispositivos sensoreados tem, além de uma aplicação prática monitorando a qualidade para as pessoas, também grande utilidade em pesquisas de construção civil e arquitetura, com medições mais precisas e incluindo um elemento muitas vezes deixado de lado: o fator humano. 

Em edifícios, escritórios são hoje os que ocupam a maior área física e tem o maior consumo de energia, sendo sistemas de iluminação, aquecimento e resfriamento (como ar condicionados) os principais causadores do alto consumo\cite{EnergyBuildings}. Por isso, escritórios são o nicho escolhido para o desenvolvimento dessa rede de dispositivos, podendo ser testada nas salas do departamento de Construção Civil ou do CICS. 


% Dificuldade em estudar elementos relacionados a conforto ?
% Falar do que já existe? 

\chapter{Descrição do Problema} % Requisitos

O conforto e a qualidade em ambientes internos é determinado através de quatro principais indicadores: \textbf{térmico, acústico, luminoso e olfativo/qualidade do ar}\cite{ComfortBox}. 

A fim de conseguirmos monitorar esses indicadores, é necessário medirmos diversos dados a respeito do ambiente a ser estudado: %%% ??  essa frase
\begin{itemize}
\item Térmico: temperatura ambiente e umidade relativa
\item Acústico: ruído ambiente
\item Luminoso: intensidade e temperatura da luz incidente
\item Qualidade do ar (e Olfativo): CO2 e VOC (\textit{volatile organic compounds})
\end{itemize}

Não apenas esses elementos são importantes, mas também a combinação deles afeta a percepção de conforto pelas pessoas \cite{ComfortOffice}. Assim, faz-se mais necessário que haja uma medição completa dos elementos presentes no ambiente a ser estudado. Mas também que essas medições estejam atreladas a opinião das pessoas a respeito do ambiente, sabendo se estão confortáveis, sendo necessário um sistema que possa coletar um \textit{feedback} das pessoas no escritório. 

%Conectividade
Todos os dados coletados, tanto das variáveis do ambiente quanto a opinião das pessoas, precisam ser salvos e disponibilizados para análise. Assim, será necessária a existência de conectividade nos dispositivos, e uma plataforma na nuvem com um banco de dados e uma interface visual para que seja feita essa análise. 

\chapter{Árvore de Objetivos} 
% arvore de objetivos

\begin{figure}[h]
%\includegraphics[scale=0.65]{objective_tree}
\includegraphics[width=\textwidth]{objective_tree}
\end{figure}

\part{Conforto em Ambientes Internos} % Tema? ~Estado da Arte~

\chapter{Indicadores de Qualidade e Conforto} %?

% Separar qualidade e conforto

% Qualidade -> Limites pela legislação e pesquisas
Como qualidade e conforto são termos subjetivos, vamos tratar aqui como "qualidade do ambiente" as condições recomendadas por normas e pesquisas, para os quatro indicadores. Isto é, será considerado um ambiente de boa qualidade o que atender às faixas de operação pré-determinadas, funcionando como um aviso para o ocupante caso as medidas indiquem que os parâmetros do ambiente estão fora do recomendado. 

Já o conforto será atrelado à percepção do usuário quanto ao ambiente. Apesar de o ambiente ser considerado saudável ou de qualidade, existem muitos fatores que afetam a sensação das pessoas, de forma que apenas a definição de uma faixa de operação não implica em bem-estar. 
% citações ??

\section{Regulamentações e Normas} %check

A legislação brasileira determina os valores máximos e mínimos dos indicadores de conforto no ambiente para que haja boas condições de trabalho: 

\begin{citacaoLonga} %Normas ministerio
\textbf{NR17 do Ministério do Trabalho} \cite{NR17}

17.5. Condições ambientais de trabalho.

17.5.2. Nos locais de trabalho onde são executadas atividades que exijam solicitação intelectual e atenção constantes, tais como: salas de controle, laboratórios, escritórios, salas de desenvolvimento ou análise de projetos, dentre outros, são recomendadas as seguintes condições de conforto:

a) níveis de ruído de acordo com o estabelecido na NBR 10152, norma brasileira registrada no INMETRO;

b) índice de temperatura efetiva entre 20oC (vinte) e 23oC (vinte e três graus centígrados);

[...]

d) umidade relativa do ar não inferior a 40 (quarenta) por cento.

17.5.2.1. Para as atividades que possuam as características definidas no subitem 17.5.2, mas não apresentam equivalência ou correlação com aquelas relacionadas na NBR 10152, o nível de ruído aceitável para efeito de conforto será de até 65 dB (A)

[...]

17.5.3.3. Os níveis mínimos de iluminamento a serem observados nos locais de trabalho são os valores de iluminâncias estabelecidos na NBR 5413, norma brasileira registrada no INMETRO.
\end{citacaoLonga}

\begin{citacaoLonga} %Normas ABNT
\textbf{NBR 10152} \cite{NBR10152} para Escritórios

Salas de reunião: 30 - 40 dB(A)

Salas de gerência, Salas de projetos e de administração: 35 - 45 dB(A)

Salas de computadores: 45 - 65 dB(A)

Salas de mecanografia: 50 - 60 dB(A)

\textbf{NBR 5413} \cite{NBR5413}

Para escritórios: 500 - 750 - 1000 lux
\end{citacaoLonga}

\section{Conforto Visual} %check
Além da \textbf{intensidade da luz incidente}, cujos níveis são estabelecida na legislação, a \textbf{temperatura da cor} da luz incidente também tem grande relevância. A muitos anos sabe-se que a luz azul emitida, de maior temperatura, causa danos à retina \cite{BlueLight}. \par
Assim, temperatura é um parâmetro importante para a qualidade do ambiente, muitas vezes deixado de lado, e assim como na saúde, afeta diretamente o conforto e a atenção das pessoas, como visto em \cite{VisualComfort}: 
\begin{itemize}
\item Conforto, luz natural: 3000K - 6000K
\item Concentração: acima de 5300K 
\end{itemize}

\section{Qualidade do Ar}
% Pesquisa CO2 e VOC
Alguns outros índices, como VOC e CO2 não são descritos na legislação. 

CO2, além de ser um gás asfixiante, perigoso em altas concentrações, também pode afetar quando em níveis moderados, na faixa de XX a XX

Segundo \cite{AirQuality}: 
- Afeta a respiração acima de 15000 ppm
- Em ambientes fechados, espera-se: 700 - 2000 ppm 

Compostos orgânicos voláteis, ou VOC (do inglês,\textit{Volatile organic compounds}, são partículas que ficam suspensas no ar, podendo vir de produtos (sintéticos ou naturais) utilizados no ambiente, como tintas, solventes, produtos de limpeza, perfumes que podem causar odor perceptível pelo ser humano\cite{AirQuality}.
Apesar de nem todos os compostos presentes no ar serem nocivos à saúde, por precaução é recomendado que eles estejam abaixo de XX \cite{}. 
Segundo \cite{AirQuality}: 
- Mean concentration  200 – 500 ug / m3
- 

\chapter{Benchmark} % ?? 
% Falar de ser Open Source

As soluções existentes hoje no mercado focam principalmente no conforto térmico ou na qualidade do ar, em sua maioria atendendo apenas dois dos indicadores de conforto apresentados. A nossa proposta é desenvolver um dispositivo que monitore de forma completa a qualidade e o conforto do ambiente, atendendo tanto aos requisitos mais comuns quanto os que não existem tanto

A conectividade do dispositivo também é um diferencial na proposta, apesar de a maioria dos dispositivos existentes hoje possuir alguma forma de conexão - no geral conexão com a internet por Wi-fi ou através de um aplicativo no smartphone, por bluetooh - os protocolos são fechados, o que dificulta a integração dos dispositivos existentes hoje entre si, com difícil escalabilidade para que seja possível ter uma integração no ambiente empresarial. 

Com um projeto Open Source, será possível não só 

% ========== Referências ==========
% --- IEEE ---
%	http://www.ctan.org/tex-archive/macros/latex/contrib/IEEEtran
%\bibliographystyle{IEEEbib}

% --- ABNT (requer ABNTeX 2) ---
%	http://www.ctan.org/tex-archive/macros/latex/contrib/abntex2
%\bibliographystyle{abntex2-num}
\bibliographystyle{plain}

\bibliography{reference_p1}


\end{document}
