\documentclass[../monografia.tex]{subfiles}
\graphicspath{ {images/}{../images/} } 

\begin{document}
% Detalhar neste item os procedimentos realizados para os testes de funcionamento dos sub-sistemas, da inter-relação dos subsistemas e do sistema completo.
% \section{Verificação dos requisitos}

% \section{Testes Realizados} %? Resultados dos Testes

\section{Validação dos Subsistemas} %? Testes Individuais
%Testes de funcionamento dos sub-sistemas

Para os sensores e o display, ao desenvolver cada uma das bibliotecas citadas na seção \ref{firmware}, também desenvolvemos exemplos para a validação de cada um individualmente. 

Utilizando a biblioteca de logging do ESP-IDF \cite{log-esp}, no modo Info, pudemos analisar os dados coletados pelos sensores, assim como os cálculos realizados, através do \textbf{IDF monitor}, uma ferramenta do ESP-IDF que funciona como um terminal serial, comunicando-se com o ESP32. 

\subsection{Intensidade da Luz - AS7262}

Para calcular a \textbf{intensidade} total da luz no ambiente fizemos alguns testes de calibração no LME, com o auxílio do técnico Carlos Ramos. Utilizando uma célula solar padrão com resposta conhecida, pudemos medir a intensidade real de uma luz branca incidindo sobre ela. Colocando em seguida o sensor AS7262, na mesma posição da célula, e incidindo a mesma luz cuja intensidade é agora conhecida, pudemos realizar medições mais precisas. 

Com leituras para diferentes intensidades, somamos os dados lidos pelo sensor para cada espectro, uma vez que o sensor tem como resposta 6 valores proporcionais à energia por unidade de área da luz incidente em cada um de seus filtros internos. 
Dividimos esse valor pelo tempo de integração configurado, cujo funcionamento foi explicado na seção \ref{as7262}, obtendo assim um valor proporcional à intensidade da luz. 	

Com esses dados, mostrados na tabela \ref{table:tabela-luz}, foi possível encontrar uma constante calibrada $\alpha$  para a conversão da leitura. \newpage

%!! tabelas do teste de luz
\begin{table}[h]
\centering
\begin{tabular}{ |c|c|c|c|c|c|c| }
	\hline
	\multicolumn{2}{|c|}{} & \multicolumn{5}{|c|}{\textbf{Intensidade da Luz} ($\mu W/cm^2$)} \\
	\hline
	\textbf{Tempo de Integração} & \textbf{Espectro} & 11000	& 5000 & 20000 & 25000 & 30000 \\
	\hline
	\hline
	\multirow{7}{3em}{84ms (0x1E)} & Violet & 6003&8167&10877&13143 & 16216 \\
	& Blue	&2451	&3368	&4512	&5484	&6822 \\
	& Green	&11271	&15485	&20939	&25654	&30721 \\
	& Yellow &13911	&19084	&25809	&30721	&30721 \\
	& Orange &3404	&4698	&6732	&7801	&9853 \\
	& Red	&1932	&2618	&3608	&4279	&5333 \\
	\hline
	& Soma	&38972	&53420	&72477	&87082	&99666 \\
	\hline
	\multirow{7}{3em}{168ms (0x3C)} & Violet & 12033&	16211&	21585&	26485&	32478\\
	&Blue	&4896	&6577	&8868	&11031	&13647 \\
	&Green	&22588	&30874	&41505	&52153	&61441 \\
	&Yellow	&27844	&38091	&51252	&61441	&61441 \\
	&Orange	&6662	&9184	&12357	&15725	&19672 \\
	&Red	&3790	&5146	&6813	&8689	&10637 \\
	\hline
	&Soma	&77813	&106083	&142380	&175524	&199316 \\
	\hline
\end{tabular}
\caption{Tabela de dados coletados na calibração do sensor AS7262}
\label{table:tabela-luz}
\end{table}
% \hfill \break

Encontramos a relação entre a soma lida pelo sensor e a intensidade da luz incidente como a constante $\alpha = \frac{Soma}{\Delta t * I} \approx 41,67$. Assim, podemos calcular os valores da intensidade lida no ambinete $I[\mu W/cm^2]=\frac{Soma}{\Delta t * \alpha}$. Como 1 lux equivale a $1,46*10^-7 W/cm^2$, temos: 

\[I[lux]= 0,146*\frac{Soma}{\Delta t * \alpha}\] %\newpage
% \hfill \break

Com esses valores de calibração, fizemos um teste mudando a luz incidente sob o sensor para validar a biblioteca completa. No dashboard é possível ver um aumento da intensidade da luz quando um abajur é ligado, em torno das 21h22, como mostra a figura \ref{fig:teste-luz}. 

\begin{figure}[h]
\centering
	\begin{subfigure}{0.5\textwidth}
		\centering
		\includegraphics[width=0.9\textwidth]{teste-luz.png}
		\caption{Arranjo do teste de luz}
		\label{fig:arranjo-luz}
	\end{subfigure}%
	\begin{subfigure}{0.5\textwidth}
		\centering
		\includegraphics[width=\textwidth]{dashboard-luz}
		\caption{Gráfico temporal das leituras}
		\label{fig:dash-luz}
	\end{subfigure}
	\caption{Testes de intensidade da luz}
	\label{fig:teste-luz}
\end{figure}
\newpage

\subsection{Qualidade do Ar - SGP30}

Para verificar o funcionamento do sensor de qualidade do ar, fizemos dois testes, monitorando os valores coletados pelo sensor através do gráfico temporal do dashboard. 

O primeiro teste foi substância em suspensão no ar. Após a calibração do sensor e uma estabilização das leituras, espirramos um ar logo acima do sensor SGP30 um odorizante de ambiente, como o mostrado na figura \ref{fig:teste-sgp30}. No dashboard em 21h07 é possível ver o aumento na concentração de TVOC e de eCO2. Esse aumento é esperado pois o odorizante é composto de substâncias orgânicas voláteis e alguns gases comprimidos que geram CO2. 
Em seguida foi ligado um ventilador de mesa ao lado do sensor. Com uma maior circulação do ar, vimos uma diminuição da concentração de TVOC e eCO2 no ambiente. 

\begin{figure}[h]
\centering
	\begin{subfigure}{0.5\textwidth}
		\centering
		\includegraphics[width=0.9\textwidth]{teste-ar}
		\caption{Arranjo do teste de suspensão}
		\label{fig:glade}
	\end{subfigure}%
	\begin{subfigure}{0.5\textwidth}
		\centering
		\includegraphics[width=0.9\textwidth]{teste-ventilador}
		\caption{Arranjo do teste com ventilador}
		\label{fig:ventilador}
	\end{subfigure}%
	\hfill
	\begin{subfigure}{0.6\textwidth}
		\centering
		\includegraphics[width=\textwidth]{dashboard-ar}
		\caption{Gráfico temporal das leituras}
		\label{fig:dash-ar}
	\end{subfigure}
	\caption{Testes de qualidade do ar}
	\label{fig:teste-sgp30}
\end{figure}
% \hfill \break

\subsection{Temperatura e Umidade - BME280}

Com um secador de cabelo, jogamos ar quente e seco sobre o sensor BME280. Pelo dashboard foi possível visualizar um aumento gradativo da temperatura e uma redução da umidade relativa do ar, após 21h15.

Com o mesmo teste feito com o ventilador \ref{fig:ventilador}, em torno de 21h10, foi possível ver que o fluxo de ar influencia também na temperatura.

\begin{figure}[h]
\centering
	\begin{subfigure}{0.5\textwidth}
		\centering
		\includegraphics[width=0.9\textwidth]{teste-secador}
		\caption{Arranjo do teste}
		\label{fig:arranjo-luz}
	\end{subfigure}%
	\begin{subfigure}{0.5\textwidth}
		\centering
		\includegraphics[width=\textwidth]{dashboard-temperatura}
		\caption{Gráfico temporal das leituras}
		\label{fig:dash-luz}
	\end{subfigure}
	\caption{Testes de temperatura e umidade}
	\label{fig:teste-sgp30}
\end{figure}
% \hfill \break

\section{Validação do Dispositivo} %? Teste Completo

Foi fabricado um case mecânico em PLA utilizando uma impressora 3D e feita a montagem de um primeiro protótipo, como mostrado na figura \ref{fig:montagem-final}, realizando os ajustes necessários no projeto para o encaixe de todas as peças. 

\begin{figure}[h]
	\centering
	\begin{subfigure}{0.5\textwidth}
		\centering
		\includegraphics[width=0.8\textwidth]{placa-final}
		\caption{Interna}
		\label{fig:interna}
	\end{subfigure}%
	\begin{subfigure}{0.5\textwidth}
		\centering
		\includegraphics[width=\textwidth]{montagem-final.jpg}
		\caption{Externa}
		\label{fig:externa}
	\end{subfigure}
	\caption{Montagem final do dispositivo}
	\label{fig:montagem-final}
\end{figure}

Para validação de cada subcircuito presente na PCB, utilizamos os mesmos exemplos citados para validação dos sensores e do display, até então testados usando uma protoboard de testes. 

Com o firmware completo desenvolvido, fizemos testes com esse dispositivo, utilizando a mesma biblioteca de Log e o IDF monitor para visualizar as leituras. Os dados dos quatro sensores operando em paralelo podem ser vistos na figura \ref{fig:monitor-sensors}.

\begin{figure}[h]
	\centering
	\includegraphics[width=0.6\textwidth]{monitor-sensors}
	\caption{Coleta dos sensores no IDF-Monitor}
	\label{fig:monitor-sensors}
\end{figure}

Cada sensor tem sua coleta sendo feita em uma task individual, isto é, a leitura e os cálculos de um sensor é feito em paralelo aos demais. Dessa forma, a ordem das mensagens sendo apresentadas no monitor terminal podem ocorrer em diferentes ordens a cada iteração, e com dados de dois sensores se intercalando, por exemplo. \newpage

Em seguida, incluímos ao código o envio dos dados via Wi-Fi, como citado na seção \ref{dev-Wi-Fi}, com o dispositivo operando como \textit{gateway}. Os dados das coletas, agora salvos no banco de dados, foram apresentados no dashboard como mostra a figura \ref{fig:dashboard-graphs}.

\begin{figure}[h]
	\centering
	\includegraphics[width=\textwidth]{dashboard-graphs}
	\caption{Gráficos de coletas do Gateway no Dashboard}
	\label{fig:dashboard-graphs}
\end{figure}

Fabricamos outros dois dispositivos para criar a rede BLE-Mesh completa, inicialmente com um dos três dispositivos atuando como "\textit{gateway}". A validação funcional de cada um deles seguiu o mesmo teste do primeiro, coletando dados do ambiente com os sensores e enviando para a plataforma via Wi-Fi. 

Com os nós da rede validados, realizando a coleta e o envio das medições do ambiente, foram feitos os testes dos mesmos nós agora utilizando a arquitetura final proposta, com Bluetooth Mesh. 

Para provisionar os dispositivos na rede Mesh, utilizamos o aplicativo nRF Mesh \cite{nrf-app}, da Nordic Semiconductor. As etapas, comuns aos três dispositivos, estão mostradas na figura \ref{fig:app-nrf-mesh}.

\begin{figure}[h]
	\centering
	\begin{subfigure}[b]{0.22\textwidth}
		\centering
		\includegraphics[width=\textwidth]{mesh-bind-1}
		\caption{Identify}
		\label{fig:mesh-bind-1}
	\end{subfigure}
	\begin{subfigure}[b]{0.22\textwidth}
		\centering
		\includegraphics[width=\textwidth]{mesh-bind-2}
		\caption{Provision}
		\label{fig:mesh-bind-2}
	\end{subfigure} %\\*
	\begin{subfigure}[b]{0.22\textwidth}
		\centering
		\includegraphics[width=\textwidth]{mesh-bind-3}
		% \caption{$y=2/x$}
		\label{fig:mesh-bind-3}
	\end{subfigure}
	\begin{subfigure}[b]{0.22\textwidth} 
		\centering
		\includegraphics[width=\textwidth]{mesh-bind-4}
		\caption{Bind key}
		\label{fig:mesh-bind-4}
	\end{subfigure}
	\caption{Telas do aplicativo nRF-Mesh}
	\label{fig:app-nrf-mesh}
\end{figure}

\newpage
Além dessas etapas, o dispositivo que atuar como gateway - nesse caso o dispositivo 1 - precisa também se inscrever (subscribe) em um grupo, esse procedimento é msotrado na figura \ref{mesh-subscribe}. Criamos um grupo com endereço 0xC100 para os testes da rede. 

\begin{figure}[h]
	\centering
	\includegraphics[scale=0.5]{mesh-subscribe.JPG}
	\caption{Tela do app nrf mesh - escolha do grupo}
	\label{mesh-subscribe}
\end{figure}

No dashboard foi possível visualizarmos os dados dos 3 dispositivos em gráficos temporais iguais aos utilizados nos testes funcionais do dispositivo (figura \ref{fig:dashboard-graphs}). Criamos também um segundo dashboard com as últimas leituras dos dispositivos, mostrado na figura \ref{fig:dashboard-medicoes}, de forma a resumir o estado atual da rede de sensores. 

\begin{figure}[h]
	\centering
	\includegraphics[width=\textwidth]{dashboard-medicoes}
	\caption{Últimas medições dos indicadores no Dashboard}
	\label{fig:dashboard-medicoes}
\end{figure}

\newpage
O dispositivo periodicamente inicia uma coleta de feedback do usuário, junto com uma medição dos indicadores do ambiente. As últimas respostas são apresentadas para o usuário como mostra a figura \ref{fig:dashboard-feedback}. 

\begin{figure}[h]
	\centering
	\includegraphics[width=\textwidth]{dashboard-feedback}
	\caption{Feedbacks no Dashboard}
	\label{fig:dashboard-feedback}
\end{figure}

Para cada parâmetro monitorado, foram definidos valores limiares segundo os níves de indicação de qualidade de ambientes para escritórios. A partir desses limiares, foram configurados alertas que são acionados caso o valor médio de certo parâmetro dentro de uma janela de 5 minutos fique acima dos valores definidos e que enviam notificações para um canal no serviço de mensagens \textit{Slack}.    

\begin{figure}[h!]
	\centering
	\includegraphics[width=\textwidth]{grafana-temperature-alert.png}
	\caption{Exemplo de ocorrência de alerta de temperatura no Dashboard}
	\label{fig:dashboard-alerta}
\end{figure}

A figura \ref{fig:dashboard-alerta} ilustra um caso em que houve o disparo de uma notificação de alerta: às 18:30, o sistema percebeu que os valores presentes nas séries de cor roxa e azul estavam acima do limiar de 31 °C, que foram monitorados até às 18:34, quando ocorreu o disparo de uma notificação de alerta, ilustrada na figura \ref{fig:slack-alerta}. Finalmente, podemos visualizar que os valores retornaram a níveis abaixo do limiar às 18:39, sendo enviada uma nova notificação para informar essa volta à normalidade, como visto na figura \ref{fig:slack-ok}.

\begin{figure}[h]
	\centering
	\begin{subfigure}[b]{0.7\textwidth}
		\includegraphics[width=\textwidth]{grafana-slack-alert.png}
		\caption{Notificação de alerta}
		\label{fig:slack-alerta}
	\end{subfigure}
	
	\begin{subfigure}[b]{0.7\textwidth}
		\includegraphics[width=\textwidth]{grafana-slack-ok.png}
		\caption{Notificação de normalidade}
		\label{fig:slack-ok}
	\end{subfigure}

	\caption{Exemplo de recebimento de notificações de alerta no serviço Slack}
	\label{fig:slack-notifications}
\end{figure}

\end{document}