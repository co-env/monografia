\documentclass[../monografia.tex]{subfiles}

\begin{document}
\section{Declaração da Necessidade}
% Contexto, PCC

Com o aumento do tempo que as pessoas passam em ambientes fechados, como escritórios, há também, nos últimos anos, um crescente interesse em monitorar e controlar tais ambientes, visando uma melhora na saúde e conforto das pessoas, e também um aumento na sua produtividade. Espaços que implementam essas soluções são comumente chamados de prédios inteligentes (\textit{smart buildings}, do inglês). É possível até mesmo que esse controle seja utilizado para uma atuação de maneira energeticamente sustentável e, dentro desse conntexto, surgem os \textit{green buildings} (em português, construções sustentáveis) \cite{GreenBuildings} \cite{EnergyBuildings}.

No desenvolvimento de construções civis sustentáveis, torna-se necessário que seja pensada a implementação automatizada deste monitoramento dos ambientes desde o projeto da edificação e sua concepção, ocorrendo de forma integrada à construção. Uma pesquisa mais aprofundada sobre o conforto dos ambientes pode interferir no projeto, de modo que sejam repensados materiais utilizados e sistemas de aquecimento, ventilação, iluminação, dentre outros. Assim como a sua implementação, pesquisas na área de conforto têm ser tornado cada vez mais importantes. 

Foi com essa necessidade e a proposta de desenvolver um dispositivo eletrônico que o professor Vanderley M. John, do departamento de Construção Civil da Poli (PCC) e coordenador do CICS (Centro de Inovação em Construção Sustentável da USP)\cite{CICS}, entrou em contato. A ideia é que seja desenvolvido um dispositivo capaz de fazer medições de parâmetros relacionados ao conforto nos ambientes internos de uma construção, coletando também a opinião das pessoas ali presentes acerca de seu bem-estar, para assim saber o real impacto dos indicadores de conforto. Para analisar todo o ambiente, é importante que existam diversos dispositivos espalhados para maior cobertura. A fim de analisar ambos os dados (medições do ambiente e opiniões), é interessante que esses dispositivos estejam conectados e integrados a uma central. 

Assim, a construção de uma rede de dispositivos sensoreados tem, além de uma aplicação prática monitorando a qualidade para as pessoas, também grande utilidade em pesquisas de construção civil e arquitetura, com medições mais precisas e incluindo um elemento muitas vezes deixado de lado: o fator humano.

Em edifícios, escritórios são hoje os que ocupam a maior área física e tem o maior consumo de energia, sendo sistemas de iluminação, aquecimento e resfriamento (como ar condicionados) os principais causadores do alto consumo\cite{EnergyBuildings}. Por isso, escritórios são o nicho escolhido para a implementação dessa rede de dispositivos, podendo ser testada nas salas do departamento de Construção Civil ou do CICS. 


% Dificuldade em estudar elementos relacionados a conforto ?
% Falar do que já existe? 

\section{Descrição do Problema} % Requisitos

O conforto e a qualidade em ambientes internos são determinados através de quatro principais indicadores: \textbf{térmico, acústico, luminoso e olfativo/qualidade do ar}\cite{ComfortBox}. 

Para que seja possível monitorar esses indicadores, é necessário medirmos diversos dados a respeito do ambiente em questão: %%% ??  essa frase
\begin{itemize}
\item Térmico: temperatura ambiente e umidade relativa
\item Acústico: ruído ambiente
\item Luminoso: intensidade e temperatura da luz incidente
\item Qualidade do ar (e Olfativo): CO2 e VOC (\textit{volatile organic compounds})
\end{itemize}

Não apenas esses elementos são importantes, mas também a combinação deles afeta a percepção de conforto pelas pessoas \cite{ComfortOffice}. Assim, faz-se mais necessário que haja uma medição completa dos elementos presentes no ambiente a ser estudado. Além disso, é interessante que essas medições estejam atreladas a opinião das pessoas a respeito do ambiente, sabendo se estão confortáveis, sendo necessário um sistema que possa coletar um \textit{feedback} das pessoas no escritório. 

%Conectividade
Todos os dados coletados, tanto das variáveis do ambiente quanto a opinião das pessoas, precisam ser salvos e disponibilizados para análise. Assim, será necessária a existência de conectividade nos dispositivos, junto de uma plataforma em nuvem com um banco de dados e uma interface visual para que seja feita essa análise. 

\section{Indicadores de Qualidade e Conforto} 

% Separar qualidade e conforto

% Qualidade -> Limites pela legislação e pesquisas
Como qualidade e conforto são termos subjetivos, vamos tratar aqui como "qualidade do ambiente" as condições recomendadas por normas e pesquisas, para os quatro indicadores. Isto é, será considerado um ambiente de boa qualidade o que atender às faixas de operação pré-determinadas, funcionando como um aviso para o ocupante caso as medidas indiquem que os parâmetros do ambiente estão fora do recomendado. 

Já o conforto será atrelado à percepção do usuário quanto ao ambiente. Apesar de o ambiente ser considerado saudável ou de qualidade, existem muitos fatores que afetam a sensação das pessoas, de forma que apenas a definição de uma faixa de operação não implica em bem-estar. 
% citações ??

\subsection{Regulamentações e Normas} %check

A legislação brasileira determina os valores máximos e mínimos dos indicadores de conforto no ambiente para que haja boas condições de trabalho: 

\begin{citacaoLonga} %Normas ministerio
\textbf{NR17 do Ministério do Trabalho} \cite{NR17}

17.5. Condições ambientais de trabalho.

17.5.2. Nos locais de trabalho onde são executadas atividades que exijam solicitação intelectual e atenção constantes, tais como: salas de controle, laboratórios, escritórios, salas de desenvolvimento ou análise de projetos, dentre outros, são recomendadas as seguintes condições de conforto:

a) níveis de ruído de acordo com o estabelecido na NBR 10152, norma brasileira registrada no INMETRO;

b) índice de temperatura efetiva entre 20oC (vinte) e 23oC (vinte e três graus centígrados);

[...]

d) umidade relativa do ar não inferior a 40 (quarenta) por cento.

17.5.2.1. Para as atividades que possuam as características definidas no subitem 17.5.2, mas não apresentam equivalência ou correlação com aquelas relacionadas na NBR 10152, o nível de ruído aceitável para efeito de conforto será de até 65 dB (A)

[...]

17.5.3.3. Os níveis mínimos de iluminamento a serem observados nos locais de trabalho são os valores de iluminâncias estabelecidos na NBR 5413, norma brasileira registrada no INMETRO.
\end{citacaoLonga}

\begin{citacaoLonga} %Normas ABNT

\textbf{NBR 10152} \cite{NBR10152} para Escritórios

Salas de reunião: 30 - 40 dB(A)

Salas de gerência, Salas de projetos e de administração: 35 - 45 dB(A)

Salas de computadores: 45 - 65 dB(A)

Salas de mecanografia: 50 - 60 dB(A)

\textbf{NBR 5413} \cite{NBR5413}

Para escritórios: 500 - 750 - 1000 lux
\end{citacaoLonga}

\subsection{Conforto Visual} %check
Além da \textbf{intensidade da luz incidente}, cujos níveis são estabelecida na legislação, a \textbf{temperatura da cor} da luz incidente também tem grande relevância. A muitos anos sabe-se que a luz azul emitida, de maior temperatura, causa danos à retina \cite{BlueLight}. \par
Assim, temperatura é um parâmetro importante para a qualidade do ambiente, muitas vezes deixado de lado, e assim como na saúde, afeta diretamente o conforto e a atenção das pessoas, como visto em \cite{VisualComfort}: 
\begin{itemize}
\item Conforto, luz natural: 3000K - 6000K
\item Concentração: acima de 5300K 
\end{itemize}

\subsection{Qualidade do Ar} % Pesquisa CO2 e VOC

% Colocar o que é IAQ, falar sobre?
% Sindrome Predios doentes? -> Intro?
Não há, na legislação brasileira, informações sobre a qualidade do ar. Isto é, concentrações de CO2 e VOC. 

A norma a seguir fala a margem esperada, mas também sem recomendações de operação. 

\begin{citacaoLonga} % ISO IAQ
\textbf{ISO 16017-2:2003}

is applicable to the measurement of airborne vapours of VOCs in a concentration range of approximately 0,002 mg/m3 to 100 mg/m3 individual organic for an exposure time of 8 h
\end{citacaoLonga}

Por conta disso, vamos nos basear em estudos que tentam relacionar as concentrações de CO2 e de VOC com efeitos na saúde e produtividade das pessoas. 

\subsubsection{CO2}
Segundo \cite{AirQuality}, o CO2 apresenta concentrações mais altas em ambientes fechados, esperando-se entre 700 e 2000 ppm, em comparação a cerca de 400ppm em ambientes abertos em áreas urbanas\cite{co2Earth}. 

O CO2, além de ser um gás asfixiante e perigoso em altas concentrações (acima de 40000ppm), também pode afetar a saúde quando em níveis moderados (abaixo de 2000ppm). 
De acordo com o \textit{Winsconsin Department of Health Services}\cite{Winsconsin}, são os efeitos causados na saúde: 
\begin{itemize}
\item 250-400ppm: Normal, concentração ambientes abertos
\item 400-1,000ppm: concentração típica em lugares fechados com pessoas, com boa circulação de ar
\item 1,000-2,000ppm: pode causar cansaço e falta de ar
\item 2,000-5,000 ppm: dores de cabeça, sonolência, e falta de ar mais intensa. Baixa concentração, perda de atenção, aumento da frequência cardíaca, e náusea
\item 40,000 ppm: Pode causar séria insuficiência respiratória, danos permanentes ao cérebro, coma e até a morte
\end{itemize}

\subsubsection{VOC}
Compostos orgânicos voláteis, ou VOC (do inglês,\textit{Volatile organic compounds}, são partículas que ficam suspensas no ar, podendo vir de produtos (sintéticos ou naturais) utilizados no ambiente, como tintas, solventes, produtos de limpeza, perfumes que podem causar odor perceptível pelo ser humano\cite{AirQuality}.

A concentração esperada é entre 0.2 e 0.5 mg/m3, ou entre 62 e 150ppb, e mesmo que nem todos os compostos presentes no ar sejam nocivos à saúde, por precaução é recomendado que estes sigam: \cite{tecam}

\begin{table}[h]
\centering
\begin{tabular}{ |c|c| }
\hline
Nível de TVOC	&   Preocupação \\
\hline
Abaixo de 93 ppb  &  Baixa \\
93 a 150 ppb  &   Pouca \\
150 a 310 ppb &  Moderada \\
310 a 930 ppb &  Alta \\
\hline
\end{tabular}
\caption{Tabela com as concentrações de TVOC e seu impacto }
\label{table}
\end{table}


A tabela feita por \cite{sensirion} mostra os níveis de TVOC aceitáveis (em ppb)

\begin{table}[t!]
\centering
\begin{tabular}{ |c|c| } 
\hline
TVOC [ppb] & Qualidade do Ar\\ 
\hline
2200 to 5500 & Insalubre \\ 
660 to 2200 & Baixa \\ 
200 to 660 & Moderada \\
65 to 220 & Boa \\
0 to 65 & Ideal \\
\hline
\end{tabular}
\caption{Tabela com as concentrações de TVOC e a correspondente qualidade do ar }
\label{table}
\end{table}

\end{document}