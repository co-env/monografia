\documentclass[../monografia.tex]{subfiles}

\graphicspath{ {images/}{../images/} }

\begin{document}

\section{Conclusões do Projeto}

O monitoramento de ambientes internos é essencial para garantir que os níveis de qualidade referentes a parâmetros gerais do ambiente estejam de acordo com os valores recomendados à saúde humana. Além disso, a coleta de valores que consideram a percepção dos ocupantes de dedterminado ambiente em relação aos parâmetros monitorados é igualmente importante, já que as pessoas passam grande parte do tempo em lugares fechados, como escritórios, e a sensação de conforto pode levar a aumentos de produtividade e satisfação no geral.

Trabalhos relacionados que têm o objetivo de monitorar ambientes dificilmente realizam medições dos quatro indicadores de qualidade definidos na seção \ref{specs-parametros}, sendo que maioria deles é focada em qualidade térmica e do ar.

A solução desenvolvida nesse projeto conseguiu realizar a medição e o monitoramento ao longo do tempo desses parâmetros e coletar opiniões sobre os mesmos periodicamente ao longo do dia, enviando-os a uma plataforma em nuvem onde podemos verificar se os parâmetros medidos estão dentro dos níveis recomendados para um escritório fechado, além de visualizar se os seus ocupantes estão confortáveis.

A rede de dispositivos se mostrou confiável, raramente apresentando alguma perda de dados. Por ser implementada utilizando Bluetooth Mesh, a rede é totalmente escalável além dos três dispositivos protótipos desenvolvidos nesse projeto, com a possibilidade de que sejam inseridos outros tipos de dispositivos que possam interagir com os do projeto. 

\section{Perspectivas de Continuidade}
\subsection{Hardware}
\subsubsection{Alimentação}

Para que os dispositivos tornem-se portáteis e independentes da rede elétrica, ao invés de alimentá-los diretamente pelo conector micro USB do módulo ESP32 seria interessante utilizar uma bateria de Lítio-Polímero de uma célula (1S). Dentre as baterias recarregáveis, a bateria LiPo possui a maior densidade energética e é uma das principais soluções utilizadas hoje em dispositivos embarcados. % do jeito que eu falei parece pedir uma referência, da pra achar depois

A bateria LiPo tem tensões de operação entre 3.5 e 4.2 Volts. Para alimentar o circuito é necessário elevar a tensão para 5V através de um regulador chaveado boost, e então regulando para 3.3V a fim de alimentar todos os subcircuitos do hardware, mantendo assim a tensão estável, com menores oscilações. 

Para fazer a recarga da bateria de forma eficiente e segura, foi pensado em um circuito carregador utilizando o CI TP4056 \cite{tp4056}, alimentado por 5V através de um conector USB-micro. Com esse CI é possível também que o circuito opere enquanto a bateria está sendo recarregada. 

\subsection{PCB}

Seria interessante desenvolver uma segunda versão da PCB, agora utilizando os CIs dos sensores e o chip ESP32, sem utilizar os módulos e kits de desenvolvimento. Isso reduz o custo da placa, além de diminuir significativamente suas dimensões. +


\subsection{Firmware}
\subsubsection{Microfone}

Para a realização do cálculo do volume ambiente em decibels é necessário ponderar o sinal recebido de forma a levar em conta a sensibilidade do ouvido. Para realizar essa filtragem correspondente, é necessário que o sinal tenha uma amostragem de ao menos 40kHz, para termos assim uma banda de 20kHz (Teorema de Nyquist). 

Como o ESP32 tem uma frequência máxima de amostragem de 6kHz, para fazer uma amostragem suficiente do microfone, seria necessário utilizar um conversor analógico-digital externo, com comunicação digital com o ESP32. Uma alternativa a isso seria trocar o microfone por um sensor microeletromecânico (MEMS, do inglês \textit{MicroElectroMechanical Sensor} com um conversor analógico-digital integrado. 

Além disso, alguns ruídos parecem influenciar mais do que o volume de forma geral, em especial ruídos de alta frequência produzidos por máquinas. Para isso, seria interessante filtrar o sinal em diferentes frequências de interesse, isto é, as que mais afetam a sensação de conforto, para conseguir identificá-las e usar a intensidade nessas frequências como indicador de conforto. 

\end{document}