\documentclass[../monografia.tex]{subfiles}

\graphicspath{ {images/}{../images/} }

\begin{document}
    Funciona :) 

\section{Futuro?}
\subsection{Harware}
\subsubsection{Alimentação}
(copiado)
Bateria de Lítio-Polímero, de uma célula (1S), por ter a maior densidade energética dentre as baterias recarregáveis, permitindo que o dispositivo seja portátil e não dependente da rede elétrica. A capacidade da bateria será definida com base nos testes de consumo do equipamento nas etapas finais de desenvolvimento do protótipo. 

A bateria LiPo tem tensões de operação entre 3.5 e 4.2 Volts. Para alimentar o circuito foi optado por elevar a tensão para 5V, através de um regulador chaveado boost, ainda não definido. 

Para fazer a recarga da bateria de forma eficiente e segura, foi pensado em um circuito carregador utilizando o CI TP4056 \cite{tp4056}, alimentado por 5V através de um conector USB-micro. Com esse CI é possível também que o circuito opere enquanto a bateria está sendo recarregada. 

\end{document}