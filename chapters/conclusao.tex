\documentclass[../monografia.tex]{subfiles}

\graphicspath{ {images/}{../images/} }

\begin{document}

\section{Conclusões do Projeto}

O monitoramento de ambientes internos é essencial para garantir que os níveis de qualidade referentes a parâmetros gerais do ambiente estejam de acordo com os valores recomendados à saúde humana. Além disso, a coleta de valores que consideram a percepção dos ocupantes de dedterminado ambiente em relação aos parâmetros monitorados é igualmente importante, já que as pessoas passam grande parte do tempo em lugares fechados, como escritórios, e a sensação de conforto pode levar a aumentos de produtividade e satisfação no geral.

Trabalhos relacionados que têm o objetivo de monitorar ambientes dificilmente realizam medições dos quatro indicadores de qualidade definidos na seção \ref{specs-parametros}, sendo que maioria deles é focada em qualidade térmica e do ar.

A solução desenvolvida nesse projeto conseguiu realizar a medição e o monitoramento ao longo do tempo desses parâmetros e coletar opiniões sobre os mesmos periodicamente ao longo do dia, enviando-os a uma plataforma em nuvem onde podemos verificar se os parâmetros medidos estão dentro dos níveis recomendados para um escritório fechado, além de visualizar se os seus ocupantes estão confortáveis.

A rede de dispositivos se mostrou confiável, raramente apresentando alguma perda de dados. Por ser implementada utilizando Bluetooth Mesh, a rede é totalmente escalável além dos três dispositivos protótipos desenvolvidos nesse projeto, com a possibilidade de que sejam inseridos outros tipos de dispositivos que possam interagir com os do projeto. 



\section{Perspectivas de Continuidade}
\subsection{Hardware}
\subsubsection{Alimentação}
(copiado)
Bateria de Lítio-Polímero, de uma célula (1S), por ter a maior densidade energética dentre as baterias recarregáveis, permitindo que o dispositivo seja portátil e não dependente da rede elétrica. A capacidade da bateria será definida com base nos testes de consumo do equipamento nas etapas finais de desenvolvimento do protótipo. 

A bateria LiPo tem tensões de operação entre 3.5 e 4.2 Volts. Para alimentar o circuito foi optado por elevar a tensão para 5V, através de um regulador chaveado boost, ainda não definido. 

Para fazer a recarga da bateria de forma eficiente e segura, foi pensado em um circuito carregador utilizando o CI TP4056 \cite{tp4056}, alimentado por 5V através de um conector USB-micro. Com esse CI é possível também que o circuito opere enquanto a bateria está sendo recarregada. 

\end{document}