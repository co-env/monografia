\documentclass[../monografia.tex]{subfiles}

\begin{document}
\section{Indicadores de Qualidade e Conforto} %?

% Separar qualidade e conforto

% Qualidade -> Limites pela legislação e pesquisas
Como qualidade e conforto são termos subjetivos, vamos tratar aqui como "qualidade do ambiente" as condições recomendadas por normas e pesquisas, para os quatro indicadores. Isto é, será considerado um ambiente de boa qualidade o que atender às faixas de operação pré-determinadas, funcionando como um aviso para o ocupante caso as medidas indiquem que os parâmetros do ambiente estão fora do recomendado. 

Já o conforto será atrelado à percepção do usuário quanto ao ambiente. Apesar de o ambiente ser considerado saudável ou de qualidade, existem muitos fatores que afetam a sensação das pessoas, de forma que apenas a definição de uma faixa de operação não implica em bem-estar. 
% citações ??

\subsection{Regulamentações e Normas} %check

A legislação brasileira determina os valores máximos e mínimos dos indicadores de conforto no ambiente para que haja boas condições de trabalho: 

\begin{citacaoLonga} %Normas ministerio
\textbf{NR17 do Ministério do Trabalho} \cite{NR17}

17.5. Condições ambientais de trabalho.

17.5.2. Nos locais de trabalho onde são executadas atividades que exijam solicitação intelectual e atenção constantes, tais como: salas de controle, laboratórios, escritórios, salas de desenvolvimento ou análise de projetos, dentre outros, são recomendadas as seguintes condições de conforto:

a) níveis de ruído de acordo com o estabelecido na NBR 10152, norma brasileira registrada no INMETRO;

b) índice de temperatura efetiva entre 20oC (vinte) e 23oC (vinte e três graus centígrados);

[...]

d) umidade relativa do ar não inferior a 40 (quarenta) por cento.

17.5.2.1. Para as atividades que possuam as características definidas no subitem 17.5.2, mas não apresentam equivalência ou correlação com aquelas relacionadas na NBR 10152, o nível de ruído aceitável para efeito de conforto será de até 65 dB (A)

[...]

17.5.3.3. Os níveis mínimos de iluminamento a serem observados nos locais de trabalho são os valores de iluminâncias estabelecidos na NBR 5413, norma brasileira registrada no INMETRO.
\end{citacaoLonga}

\begin{citacaoLonga} %Normas ABNT

\textbf{NBR 10152} \cite{NBR10152} para Escritórios

Salas de reunião: 30 - 40 dB(A)

Salas de gerência, Salas de projetos e de administração: 35 - 45 dB(A)

Salas de computadores: 45 - 65 dB(A)

Salas de mecanografia: 50 - 60 dB(A)

\textbf{NBR 5413} \cite{NBR5413}

Para escritórios: 500 - 750 - 1000 lux
\end{citacaoLonga}

\subsection{Conforto Visual} %check
Além da \textbf{intensidade da luz incidente}, cujos níveis são estabelecida na legislação, a \textbf{temperatura da cor} da luz incidente também tem grande relevância. A muitos anos sabe-se que a luz azul emitida, de maior temperatura, causa danos à retina \cite{BlueLight}. \par
Assim, temperatura é um parâmetro importante para a qualidade do ambiente, muitas vezes deixado de lado, e assim como na saúde, afeta diretamente o conforto e a atenção das pessoas, como visto em \cite{VisualComfort}: 
\begin{itemize}
\item Conforto, luz natural: 3000K - 6000K
\item Concentração: acima de 5300K 
\end{itemize}

\subsection{Qualidade do Ar} % Pesquisa CO2 e VOC

% Colocar o que é IAQ, falar sobre?
% Sindrome Predios doentes? -> Intro?
Não há, na legislação brasileira, informações sobre a qualidade do ar. Isto é, concentrações de CO2 e VOC. 

A norma a seguir fala a margem esperada, mas também sem recomendações de operação. 

\begin{citacaoLonga} % ISO IAQ
\textbf{ISO 16017-2:2003}

is applicable to the measurement of airborne vapours of VOCs in a concentration range of approximately 0,002 mg/m3 to 100 mg/m3 individual organic for an exposure time of 8 h
\end{citacaoLonga}

Por conta disso, vamos nos basear em estudos que tentam relacionar as concentrações de CO2 e de VOC com efeitos na saúde e produtividade das pessoas. 

\subsubsection{CO2}
Segundo \cite{AirQuality}, o CO2 apresenta concentrações mais altas em ambientes fechados, esperando-se entre 700 e 2000 ppm, em comparação a cerca de 400ppm em ambientes abertos em áreas urbanas\cite{co2Earth}. 

O CO2, além de ser um gás asfixiante e perigoso em altas concentrações (acima de 40000ppm), também pode afetar a saúde quando em níveis moderados (abaixo de 2000ppm). 
De acordo com o \textit{Winsconsin Department of Health Services}\cite{Winsconsin}, são os efeitos causados na saúde: 
\begin{itemize}
\item 250-400ppm: Normal, concentração ambientes abertos
\item 400-1,000ppm: concentração típica em lugares fechados com pessoas, com boa circulação de ar
\item 1,000-2,000ppm: pode causar cansaço e falta de ar
\item 2,000-5,000 ppm: dores de cabeça, sonolência, e falta de ar mais intensa. Baixa concentração, perda de atenção, aumento da frequência cardíaca, e náusea
\item 40,000 ppm: Pode causar séria insuficiência respiratória, danos permanentes ao cérebro, coma e até a morte
\end{itemize}

\subsubsection{VOC}
Compostos orgânicos voláteis, ou VOC (do inglês,\textit{Volatile organic compounds}, são partículas que ficam suspensas no ar, podendo vir de produtos (sintéticos ou naturais) utilizados no ambiente, como tintas, solventes, produtos de limpeza, perfumes que podem causar odor perceptível pelo ser humano\cite{AirQuality}.

A concentração esperada é entre 0.2 e 0.5 mg/m3, e mesmo que nem todos os compostos presentes no ar sejam nocivos à saúde, por precaução é recomendado que estes sigam: \cite{tecam}

\begin{tabular}{ |c|c| }
\hline
TVOC Level [mg/m3]	&   Level of Concern \\
\hline
Less than 0.3 mg/m3	 &  Low \\
0.3 to 0.5 mg/m3	&   Acceptable \\
0.5 to 1 mg/m3	 &  Marginal \\
1 to 3 mg/m3	 &  High \\
\hline
\end{tabular}

%A tabela feita por \cite{sensirion} mostra os níveis de TVOC aceitáveis (em ppb)
%\begin{tabular}{ |c|c| } 
%\hline
%TVOC [ppb] & Hygienic Rating\\ 
%\hline
%2200 to 5500 & Unhealty \\ 
%660 to 2200 & Poor \\ 
%200 to 660 & Moderate \\
%65 to 220 & Good \\
%0 to 65 & Excellent \\
%\hline
%\end{tabular}
%1 ppb = 1 mg/m3


\section{Soluções de Monitoramento} % ?? 
% Falar de ser Open Source

A tabela a seguir mostra algumas das principais soluções encontradas no mercado para monitoramento de ambientes fechados: 

\begin{center}
\begin{tabular}{ | m{2.5cm} | m{2.4cm}| m{2.2cm} |m{2cm} |m{2.1cm} |m{2.8cm} | } 
\hline
\textbf{Projeto} & \textbf{Térmico} & \textbf{Luminoso} & \textbf{Acústico} & \textbf{Ar} & \textbf{Conectividade} \\ 
\hline
Multi comfort \cite{multicomfort} & Sim* & Sim* & Sim* & Sim* & Sim* \\ \hline
MC350\cite{mc350} & Sim* & Sim* & Sim* & & Bluetooth, App \\
\hline
Metriful Sense\cite{metriful} & Temperatura, Umidade, Pressão & Intensidade & Volume, Frequência & VOC &  \\ \hline
CoMoS\cite{CoMoS} & Temperatura, Umidade, Veloc Ar & Intensidade & & & Wifi, SW Web \\ \hline
HC tech\cite{HCTech} & Temperatura, Umidade & Intensidade & & & Sigfox, SW Web \\ \hline
ECOMLITE \cite{ECOMLITE} & Temperatura, Umidade, Pressão & & Volume & CO2, VOC, CO, NO2 & Wi-fi, Zigbee, Ethernet, SW Web \\ \hline
Netatmo\cite{netatmo} & Temperatura, Umidade, & & Volume & CO2 & Wi-fi, App \\ \hline
Senlab O\cite{Senlab} & Temperatura, Umidade & Intensidade & & & LoRa \\ \hline
Comfort Click\cite{comfortclick} & Temperatura, Umidade & & Volume & & Wi-fi, App  \\ \hline
\end{tabular}
\end{center}

\begin{flushright}
(*) Solução não detalhada pela construtora
\end{flushright}


% Análise  - Finalizar
Ao observar as soluções existentes no mercado, é possível ver que a maioria atende a apenas alguns dos indicadores. 

Multi Comfort \cite{multicomfort}, a solução mais completa encontrada, trata-se de uma sistema desenvolvido pela construtora Saint-Gobain de forma integrada à construção do edifício. Como apresentado pelo prof. Vanderley (PCC), o monitoramento e automação na construção civil vem crescendo nos últimos anos, de forma que soluções desse tipo tendem a tornar-se mais comuns. 
Apesar de atender aos 4 indicadores de qualidade e conforto, essa solução não apresenta detalhes sobre as medições que são feitas, quais os elementos medidos ou a precisão dessas medições, assim como sobre a sua conectividade. Também apresenta grande dificuldade para integração em um ambiente já construído. Nesse contexto, é introduzido o equipamento MC350\cite{mc350} - da mesma empresa -, mas que não atende mais a todos os indicadores. 

Em seguida, temos o Metriful Sense \cite{metriful}, que também atende a todos os indicadores de conforto, que é um produto novo em \textit{crowdfunding}. Diferentemente das demais, esse não é um dispositivo completo, mas uma plataforma de sensores projetada para o monitoramento de ambientes, que necessita de uma interface com outro processador e possivelmente com um módulo \textit{wireless} para que haja conexão com uma plataforma de análise. 

Os demais produtos atendem em média apenas dois dos indicadores de conforto, sendo apenas o conforto térmico presente em todos. O conforto luminoso, quando presente, é atendido de forma superficial, sendo medida apenas a intensidade da luz e não a temperatura, que é um elemento importante na sensação de conforto \cite{VisualComfort}. Já no monitoramento da qualidade do ar, apenas uma das soluções \cite{ECOMLITE} (cuja especificação está disponível) mede ao menos os dois principais elementos indicadores \cite{AirQuality}. 

Vemos, assim, que nenhum dos produtos existentes no mercado atende a todos os requisitos levantados para a nossa aplicação. A proposta aqui é o desenvolvimento de um dispositivo que monitore a qualidade e o conforto do ambiente atendendo a todos os indicadores apresentados de forma completa, e com uma medição mais aprofundada em conforto luminoso e qualidade do ar que os projetos comparados. 

Além disso, nenhuma das soluções encontradas envolve diretamente a opinião das pessoas frequentando o ambiente na análise dos seus dados, apenas no controle do ambiente quando esse não atende os requisitos de qualidade ou de conforto. 

Vale destacar, por fim, que na maioria dos casos essas soluções são fechadas, com protocolos que dificultam a integração dos dispositivos com outros que possam complementá-los em seu funcionamento, ou de forma que seja possível uma centralização da análise dos dados coletados. Desenvolvendo um projeto \textit{open-source} e com protocolos padrão de comunicação sem fio, isso deixa de ser um problema. 


\end{document}